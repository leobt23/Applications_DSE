\section{Exploratory Data Analysis}

\subsection{European cardholders's transactions}

This dataset represents credit card transactions made by European cardholders in September 2013 over a span of two days.

Out of the 284,807 transactions recorded, 492 were fraudulent, making up a mere 0.172\% of the total transactions. Given the disparity in numbers, the dataset is notably imbalanced.

The dataset consists mainly of numerical variables derived from a Principal Component Analysis (PCA) transformation. To maintain confidentiality, the original features and further details about the data are not disclosed. The variables V1, V2, through V28 are the principal components derived from the PCA. The only exceptions that haven't undergone PCA are 'Time' and 'Amount'. The 'Time' feature indicates the seconds that have passed between each transaction and the first one in the dataset. On the other hand, 'Amount' denotes the transaction value, which can be useful for context-sensitive learning approaches. The 'Class' feature is the target variable, where a value of 1 indicates a fraudulent transaction, and 0 signifies a legitimate one.

There is no null or missing values of the dataset, but was 1854 observations duplicated, which were removed because they don't add any value to the analysis.

\begin{table}[htbp]
    \caption{Summary Statistics for Fraud and Valid Transactions}
    \begin{center}
        \begin{tabular}{|c|c|c|}
            \hline
            \textbf{Statistic}     & \textbf{Fraud} & \textbf{Valid} \\
            \hline
            Count                  & 492            & 284315         \\
            \hline
            Mean                   & 122.21         & 88.29          \\
            \hline
            Standard Deviation     & 256.68         & 250.11         \\
            \hline
            Minimum                & 0.0            & 0.0            \\
            \hline
            25\% Quantile          & 1.0            & 5.65           \\
            \hline
            50\% Quantile (Median) & 9.25           & 22.0           \\
            \hline
            75\% Quantile          & 105.89         & 77.05          \\
            \hline
            Maximum                & 2125.87        & 25691.16       \\
            \hline
        \end{tabular}
        \label{tab:transaction_summary}
    \end{center}
\end{table}

\subsection{Analysis of Transaction Amounts for Fraudulent and Valid Transactions}

The table presents a summary of transaction amounts for both fraudulent and valid categories. A notable observation is the disparity in the count of transactions: while there are only 492 fraudulent transactions, there are a staggering 284,315 valid ones. This vast difference highlights the class imbalance inherent in the dataset.

When examining the transaction amounts, fraudulent transactions have a mean value of approximately \$122.21, which is slightly higher than the valid transactions' mean of \$88.29. Despite this, the maximum fraudulent transaction (\$2125.87) is significantly lower than the highest valid transaction, which reaches a substantial \$25,691.16.

The spread of transaction amounts, as represented by the standard deviation, is roughly similar for both classes, hovering around the \$250 mark. The median value for valid transactions (\$22.00) is more than double that of fraudulent transactions (\$9.25), indicating that half of the fraudulent transactions are below this amount.

While fraudulent transactions are comparatively rarer, their amounts can vary significantly, sometimes even surpassing the typical amounts seen in valid transactions. It underscores the importance of leveraging advanced techniques to detect such anomalies amidst the vast sea of valid transactions.

\sub

\subsubsection{Imbalanced Dataset}

An imbalanced (or unbalanced) dataset refers to a situation where the number of observations belonging to one class is significantly lower than those belonging to the other classes. In the context of a binary classification, which consists of two classes, an imbalanced dataset typically has a disproportionate ratio of observations in one class compared to the other.
This problem can lead to:
\begin{itemize}
    \item \textbf{Performance Deception:} Traditional metrics like accuracy can be misleading. A naive classifier predicting only the majority class would still achieve a very high accuracy due to the imbalance.
    \item \textbf{Model Bias:} Many machine learning models might exhibit a bias towards the majority class, often ignoring the minority class.
    \item \textbf{Decreased Predictive Power:} The minority class, in this case, the class of higher interest might not be predicted well, leading to a higher number of false negatives.
\end{itemize}

Given that our dataset has a class with 99.83\% representation, it's highly imbalanced. I will present later on this report some considerations and how to deal with this problem.

subsubsection{Correlation Matrix}
The fact the dataset is unbalanced can lead to misleading results when using correlation metrics:
\begin{itemize}
    \item In highly imbalanced datasets, even small patterns in the minority class can result in seemingly strong correlations. This can lead to the model finding relationships that aren't generalizable.
    \item Correlation measures in an imbalanced dataset might be dominated by the majority class, potentially overshadowing meaningful relationships present in the minority class.
    \item Also, because the minority class has fewer data points, there's a higher risk of not detecting a relationship (false negative) when one might exist.
\end{itemize}

Due to the problem I presented, Traditional correlation metrics may not be suitable for imbalanced datasets. In this binary classification problem with a highly imbalanced target, a high Pearson correlation coefficient with a predictor might be misleading.

\subsubsection{Distributions}

Due to the PCA transformation, the features are not directly interpretable. However, we can still examine the distributions of the features to gain some insights.
The distributions of the features for both fraudulent and valid transactions are shown in Figure \ref{eda_distributions_all_fraud_or_not}.
The goal was to understand the distribution of the features for both classes.
Several features exhibit distinct distribution patterns when comparing the two Class categories: V4 and V11 show clear segregation between Class 0 and Class 1, suggesting high discriminatory power.
Features V12, V14, and V18 display partial separation, indicating moderate distinction between classes. V1, V2, V3, and V10 each present a somewhat unique distribution, whereas V25, V26, and V28 appear to have overlapping distributions for both classes.
Generally speaking, except for the 'Time' and 'Amount' features, the distribution of features for genuine transactions (Class = 0) tends to be centralized around 0, often with an extended tail at one end.
Conversely, distributions for fraudulent transactions (Class = 1) are typically asymmetrical, reflecting a deviation from the norm.



\begin{figure}[htbp]
    \centerline{\includegraphics[scale=0.5]{images/initial_dataset_balance.png}}
    \caption{Example of a figure caption.}
    \label{initial_dataset_balance}
\end{figure}


\subsection{Variance}


The majority of the data is the result of a PCA transformation - in this case was to \textbf{guarantee anonymity of the clients} - several unique characteristics and considerations come into play:

\begin{enumerate}
    \item \textbf{Orthogonality:} The principal components resulting from PCA are orthogonal (uncorrelated). Thus, the correlation matrix of these components should be a diagonal matrix with ones on the diagonal (or very close to this in practice due to numerical precision).

    \item \textbf{Variance Explained:} One of the key aspects of PCA is the amount of variance explained by each principal component. The first few components typically capture the majority of the variance in the dataset, while the latter components capture less and less variance.
\end{enumerate}

Given these considerations, the variation analysis that makes the most sense for PCA-transformed data includes:

\begin{enumerate}
    \item \textbf{Variance Explained:}
          \begin{itemize}
              \item \textbf{Scree Plot:} A plot showing the fraction of total variance explained by each principal component. This helps in determining how many components to retain for further analysis.
          \end{itemize}
\end{enumerate}

PCA is a dimensionality reduction technique that seeks to identify axes in data that maximize variance. The method involves computing the eigenvalues and eigenvectors of the dataset's covariance matrix. The eigenvectors, termed principal components, determine the direction of the new feature space, while the eigenvalues define their magnitude, i.e., the variance in those directions.


\begin{enumerate}
    \item \textbf{Eigenvalues and Variance}: The eigenvalue associated with each principal component signifies the variance along that component. In PCA, these components are ordered by descending eigenvalues. This means the first principal component encapsulates the largest variance in the dataset.

    \item \textbf{Orthogonality}: Every principal component is orthogonal to every other, implying they are uncorrelated. Hence, each subsequent component captures the direction of maximum variance not represented by the preceding components.

    \item \textbf{Maximizing Variance}: The first principal component (often denoted as \( PC1 \)) represents the direction in the original feature space capturing the utmost variance. The second principal component (often \( PC2 \)) is orthogonal to \( PC1 \) and represents the second-highest variance, and so forth.
\end{enumerate}

Consequently, when PCA-transformed features are acquired (like 'V1', 'V2', 'V3', etc.), they are inherently ordered by the variance they represent from the original dataset, with 'V1' representing the most and the final component the least.





