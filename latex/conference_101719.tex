\documentclass[conference]{IEEEtran}
\IEEEoverridecommandlockouts
% The preceding line is only needed to identify funding in the first footnote. If that is unneeded, please comment it out.
\usepackage{cite}
\usepackage{amsmath,amssymb,amsfonts}
\usepackage{algorithmic}
\usepackage{graphicx}
\usepackage{textcomp}
\usepackage{xcolor}
\def\BibTeX{{\rm B\kern-.05em{\sc i\kern-.025em b}\kern-.08em
    T\kern-.1667em\lower.7ex\hbox{E}\kern-.125emX}}
\begin{document}

\title{Benchmarking anomaly detection techniques in the fraud detection domain*\\
{\footnotesize \textsuperscript{*}Note: Sub-titles are not captured in Xplore and
should not be used}
\thanks{Identify applicable funding agency here. If none, delete this.}
}

\author{\IEEEauthorblockN{1\textsuperscript{st} Leonardo Brito}
    \IEEEauthorblockA{\textit{Engineering and Data Science} \\
        \textit{Instituto Superior Técnico}\\
        Lisbon, Portugal \\
        leonardo.amado.brito@tecnico.ulisboa.pt}
    \and
    \IEEEauthorblockN{2\textsuperscript{nd} Given Name Surname}
    \IEEEauthorblockA{\textit{dept. name of organization (of Aff.)} \\
        \textit{name of organization (of Aff.)}\\
        City, Country \\
        email address or ORCID}
}

\maketitle

\begin{abstract}
    This document is a model and instructions for \LaTeX.
    This and the IEEEtran.cls file define the components of your paper [title, text, heads, etc.]. *CRITICAL: Do Not Use Symbols, Special Characters, Footnotes,
    or Math in Paper Title or Abstract.
\end{abstract}

\begin{IEEEkeywords}
    component, formatting, style, styling, insert
\end{IEEEkeywords}

\section{Introduction}
In the realm of digital transactions, the significance of detecting fraud at its earliest is paramount. Supervised learning models are often the preferred choice, renowned for their accuracy, but they hinge on the availability of labeled data. This dependency can introduce operational delays, leaving a window where transactions might be exposed to unchecked fraudulent activities.

To address this gap, there's a growing interest in leveraging unsupervised learning. Without the need for labeled data, unsupervised anomaly detection techniques offer a promising avenue, especially when one considers that fraud typically surfaces as statistical anomalies amidst legitimate transactions.

Drawing from open-source datasets, this report sets out to critically evaluate a range of unsupervised anomaly detection methods in the context of fraud detection. Our approach is methodical: starting with a detailed data exploration, we then benchmark against a supervised baseline, and finally, analyze the performance of both classical and deep learning unsupervised models. The overarching aim is to ascertain the potential of unsupervised techniques in offering interim protection against fraud during periods when supervised models are not yet feasible.

\section{Data Exploration}

\subsection{European cardholders's transactions}

This dataset represents credit card transactions made by European cardholders in September 2013 over a span of two days.

Out of the 284,807 transactions recorded, 492 were fraudulent, making up a mere 0.172% of the total transactions. Given the disparity in numbers, the dataset is notably imbalanced.

The dataset consists mainly of numerical variables derived from a Principal Component Analysis (PCA) transformation. To maintain confidentiality, the original features and further details about the data are not disclosed. The variables V1, V2, through V28 are the principal components derived from the PCA. The only exceptions that haven't undergone PCA are 'Time' and 'Amount'. The 'Time' feature indicates the seconds that have passed between each transaction and the first one in the dataset. On the other hand, 'Amount' denotes the transaction value, which can be useful for context-sensitive learning approaches. The 'Class' feature is the target variable, where a value of 1 indicates a fraudulent transaction, and 0 signifies a legitimate one.

\begin{figure}[htbp]
    \centerline{\includegraphics[scale=0.5]{images/initial_dataset_balance.png}}
    \caption{Example of a figure caption.}
    \label{initial_dataset_balance}
\end{figure}

\begin{table}[htbp]
    \caption{Table Type Styles}
    \begin{center}
        \begin{tabular}{|c|c|c|c|}
            \hline
            \textbf{Table} & \multicolumn{3}{|c|}{\textbf{Table Column Head}}                                                         \\
            \cline{2-4}
            \textbf{Head}  & \textbf{\textit{Table column subhead}}           & \textbf{\textit{Subhead}} & \textbf{\textit{Subhead}} \\
            \hline
            copy           & More table copy$^{\mathrm{a}}$                   &                           &                           \\
            \hline
            \multicolumn{4}{l}{$^{\mathrm{a}}$Sample of a Table footnote.}
        \end{tabular}
        \label{tab1}
    \end{center}
\end{table}

\begin{figure}[htbp]
    \centerline{\includegraphics{fig1.png}}
    \caption{Example of a figure caption.}
    \label{fig}
\end{figure}


\section*{References}

Please number citations consecutively within brackets \cite{b1}.

\begin{thebibliography}{00}
    \bibitem{b1} G. Eason, B. Noble, and I. N. Sneddon, ``On certain integrals of Lipschitz-Hankel type involving products of Bessel functions,'' Phil. Trans. Roy. Soc. London, vol. A247, pp. 529--551, April 1955.
    \bibitem{b2} J. Clerk Maxwell, A Treatise on Electricity and Magnetism, 3rd ed., vol. 2. Oxford: Clarendon, 1892, pp.68--73.
    \bibitem{b3} I. S. Jacobs and C. P. Bean, ``Fine particles, thin films and exchange anisotropy,'' in Magnetism, vol. III, G. T. Rado and H. Suhl, Eds. New York: Academic, 1963, pp. 271--350.
    \bibitem{b4} K. Elissa, ``Title of paper if known,'' unpublished.
    \bibitem{b5} R. Nicole, ``Title of paper with only first word capitalized,'' J. Name Stand. Abbrev., in press.
    \bibitem{b6} Y. Yorozu, M. Hirano, K. Oka, and Y. Tagawa, ``Electron spectroscopy studies on magneto-optical media and plastic substrate interface,'' IEEE Transl. J. Magn. Japan, vol. 2, pp. 740--741, August 1987 [Digests 9th Annual Conf. Magnetics Japan, p. 301, 1982].
    \bibitem{b7} M. Young, The Technical Writer's Handbook. Mill Valley, CA: University Science, 1989.
\end{thebibliography}
\vspace{12pt}
\color{red}
IEEE conference templates contain guidance text for composing and formatting conference papers. Please ensure that all template text is removed from your conference paper prior to submission to the conference. Failure to remove the template text from your paper may result in your paper not being published.

\end{document}
